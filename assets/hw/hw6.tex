\documentclass[12pt]{article}\usepackage[]{graphicx}\usepackage[]{color}
%% maxwidth is the original width if it is less than linewidth
%% otherwise use linewidth (to make sure the graphics do not exceed the margin)
\makeatletter
\def\maxwidth{ %
  \ifdim\Gin@nat@width>\linewidth
    \linewidth
  \else
    \Gin@nat@width
  \fi
}
\makeatother

\definecolor{fgcolor}{rgb}{0.345, 0.345, 0.345}
\newcommand{\hlnum}[1]{\textcolor[rgb]{0.686,0.059,0.569}{#1}}%
\newcommand{\hlstr}[1]{\textcolor[rgb]{0.192,0.494,0.8}{#1}}%
\newcommand{\hlcom}[1]{\textcolor[rgb]{0.678,0.584,0.686}{\textit{#1}}}%
\newcommand{\hlopt}[1]{\textcolor[rgb]{0,0,0}{#1}}%
\newcommand{\hlstd}[1]{\textcolor[rgb]{0.345,0.345,0.345}{#1}}%
\newcommand{\hlkwa}[1]{\textcolor[rgb]{0.161,0.373,0.58}{\textbf{#1}}}%
\newcommand{\hlkwb}[1]{\textcolor[rgb]{0.69,0.353,0.396}{#1}}%
\newcommand{\hlkwc}[1]{\textcolor[rgb]{0.333,0.667,0.333}{#1}}%
\newcommand{\hlkwd}[1]{\textcolor[rgb]{0.737,0.353,0.396}{\textbf{#1}}}%
\let\hlipl\hlkwb

\usepackage{framed}
\makeatletter
\newenvironment{kframe}{%
 \def\at@end@of@kframe{}%
 \ifinner\ifhmode%
  \def\at@end@of@kframe{\end{minipage}}%
  \begin{minipage}{\columnwidth}%
 \fi\fi%
 \def\FrameCommand##1{\hskip\@totalleftmargin \hskip-\fboxsep
 \colorbox{shadecolor}{##1}\hskip-\fboxsep
     % There is no \\@totalrightmargin, so:
     \hskip-\linewidth \hskip-\@totalleftmargin \hskip\columnwidth}%
 \MakeFramed {\advance\hsize-\width
   \@totalleftmargin\z@ \linewidth\hsize
   \@setminipage}}%
 {\par\unskip\endMakeFramed%
 \at@end@of@kframe}
\makeatother

\definecolor{shadecolor}{rgb}{.97, .97, .97}
\definecolor{messagecolor}{rgb}{0, 0, 0}
\definecolor{warningcolor}{rgb}{1, 0, 1}
\definecolor{errorcolor}{rgb}{1, 0, 0}
\newenvironment{knitrout}{}{} % an empty environment to be redefined in TeX

\usepackage{alltt}
\usepackage{fullpage,enumitem,amsmath,amssymb,amsthm,graphicx}
\usepackage{hyperref}
\usepackage{scrextend}
\usepackage{booktabs}

\newcommand{\ba}{\color{blue}\begin{addmargin}[2em]{0em}}
\newcommand{\ea}{\end{addmargin}\color{black}}

\newcommand{\be}{\begin{eqnarray}}
\newcommand{\ee}{\end{eqnarray}}

\newcommand{\pder}[2]{\frac{\partial #1}{\partial #2}}
\newcommand{\E}{\mathbb{E}}



\IfFileExists{upquote.sty}{\usepackage{upquote}}{}
\begin{document}

\begin{center}
{\Large Talis Biomedical Statistics Course - Homework 6} \\
{\small \textbf{Due: 22 January 2020 9:00 AM}}
\newline

\begin{tabular}{rl}
  Name: & [your first and last name] \\
  Collaborators: & [list all the people you worked with] \\
  Date: & [date of submission]
\end{tabular}
\end{center}

By turning in this assignment, I agree by the \textbf{\href{https://communitystandards.stanford.edu/policies-and-guidance/honor-code}{Stanford honor code}} and declare
that all of this is my own work. \\

\vspace{.2in}

The exercises can be found in the DeGroot \& Schervish's textbook, available
\textcolor{blue}{\href{http://professor.ufabc.edu.br/~nelson.faustino/Ensino/IPE2016/Livros/Morris\%20H\%20DeGroot_\%20Mark\%20J\%20Schervish-Probability\%20and\%20statistics-Pearson\%20Education\%20\%20(2012)\%20(1).pdf}{here}}.


%%%%%%%%%%%%%%%%%%%%%%%%%%%%%%%%%%%%%%%%
\subsection*{Problem 1}

Section 7.5 Exercise 2 \\


%%%%%%%%%%%%%%%%%%%%%%%%%%%%%%%%%%%%%%%%
\subsection*{Problem 2}

Section 7.5 Exercise 3 \\


%%%%%%%%%%%%%%%%%%%%%%%%%%%%%%%%%%%%%%%%
\subsection*{Problem 3}

Section 7.5 Exercise 6 \\


%%%%%%%%%%%%%%%%%%%%%%%%%%%%%%%%%%%%%%%%
\subsection*{Problem 4}

\begin{enumerate}[label=(\alph*)]
  \item Section 7.5 Exercise 9

  \item Suppose that $X_1, \ldots, X_n$ form a random sample from a distribution
  with the pdf given in Part (a). Find the MLE of $e^{-1/\theta}$. \textit{n.b.
  you'll want to read Section 7.6 before answering this part.}

\end{enumerate}


%%%%%%%%%%%%%%%%%%%%%%%%%%%%%%%%%%%%%%%%
\subsection*{Problem 5}

Section 7.6 Exercise 3 \\


%%%%%%%%%%%%%%%%%%%%%%%%%%%%%%%%%%%%%%%%
\subsection*{Problem 6}

Section 7.6 Exercise 4 \\


\end{document}
